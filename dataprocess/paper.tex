\documentclass[11pt]{article}
\usepackage[top=1in, bottom=1in, left=1in, right=1in]{geometry}
\usepackage{tikz}
\usetikzlibrary{shapes,arrows,matrix,patterns,positioning}
\usepackage{color}
%\usepackage[parfill]{parskip}
\usepackage{graphicx}
\usepackage{algorithmic}
\usepackage{amssymb}
\usepackage{epstopdf}
\usepackage[ruled]{algorithm2e}
\usepackage{float}
\usepackage{amsmath,amsthm,bm,color,epsfig,enumerate,caption}
\usepackage{hyperref}
\usepackage{booktabs}
\usepackage{multirow}
\usepackage{array}

%\linespread{2}

% to highlight the latest changes
\newcommand{\alert}[1]{\textcolor{red}{#1}}

\newcommand{\hz}[1]{{\textbf{HZ: #1}}}
\newcommand{\yl}[1]{\textcolor{orange}{\textbf{Ryan: #1}}}
\newcommand{\erm}[1]{\textcolor{red}{\textbf{EM: #1}}}
\newtheorem{theorem}{Theorem}[section]
\newtheorem{outline}[theorem]{Outline}
\newtheorem{corollary}[theorem]{Corollary}
\newtheorem{definition}[theorem]{Definition}
\newtheorem{lemma}[theorem]{Lemma}
\newtheorem{proposition}[theorem]{Proposition}
\newtheorem{algo}[theorem]{Algorithm}
\newtheorem{remark}[theorem]{Remark}
\renewcommand{\appendix}[1]{
\section*{Appendix: #1}
}
\newcommand{\norm}[1]{\left\lVert#1\right\rVert}
%\newcommand{\herm}[1]{(#1)^*}
%\renewcommand{\L}{\mathcal{L}}
%\renewcommand{\O}{\mathcal{O}}
\renewcommand{\O}{O}
%\newcommand{\K}{\mathcal{K}}
%\newcommand{\F}{\mathcal{F}}
\newcommand{\bbZ}{\mathbb{Z}}
\newcommand{\bbF}{\mathbb{F}}
\newcommand{\bbC}{\mathbb{C}}
\newcommand{\bbR}{\mathbb{R}}
\newcommand{\rID}{{\it rID}}
\newcommand{\cID}{{\it cID}}
\newcommand{\blue}{\textcolor{blue}}

\makeatletter
\newcommand*{\extendadd}{
  \mathbin{
    \mathpalette\extend@add{}
  }
}
\newcommand*{\extend@add}[2]{
  \ooalign{
    $\m@th#1\leftrightarrow$%
    \vphantom{$\m@th#1\updownarrow$}
    \cr
    \hfil$\m@th#1\updownarrow$\hfil
  }
}
\makeatother

\begin{document}

\title{Numerical Comparison Summary}
%\author{Qiyuan Pang \\ Tsinghua University, China\\  \href{mailto:ppangqqyz@foxmail.com}{ppangqqyz@foxmail.com} 
%   \and Kenneth L. Ho \\ San Francisco, CA, USA\\ \href{mailto:klho@alumni.caltech.edu}{klho@alumni.caltech.edu}
%     \and Haizhao Yang \\ Department of Mathematics\\ National University of Singapore, Singapore\\ \href{mailto:haizhao@nus.edu.sg}{haizhao@nus.edu.sg} }
\author{Qiyuan Pang}

\maketitle

The following quantities are used in the rest of the section to
evaluate the performance of the preconditioner:

\begin{itemize}
\item N: problem size;
\item $e_{a}$: the relative error set for the butterfly approximation $\hat{K}$ of $K$;
\item $\epsilon$: the fixed tolerance set in HIF/HQR;
\item $e_{f}$: forward error of a factorization (HIF/HQR) (e.g., $\hat{A}$ factorizes $A$, then $e_{f} = \|\hat{A}x-Ax\|/\|Ax\|$);
\item $e_{h}$: the accuracy of HODLR construction using the peeling algorithm.
%\item $r_{h}$: the maximum rank recorded from the HODLR construction above.
\item $e_{s}$: the relative error of the approximation $\hat{G}\hat{K}^{*}$ of $K^{-1}$, defined as $\|\hat{G}\hat{K}^{*}b - x\|/\|x\|$ where $x$ is a random vector and $b = Kx$;
\item $n_{i}$: the number of iterations used in PCG until covergence;
\item $e$: the relative error of the solution returned by PCG.
\end{itemize}

Among all experiments below, the stopping criteria set for PCG is tolerance $1e-10$.

\textbf{Examples (1D).} We begin with an example of 1D discrete FIO of the form
\begin{equation*}
u(x) = \int\limits_{\mathbb{R}}a(x)e^{2\pi i \Phi(x, \xi)}\hat{f}(\xi) d\xi
\end{equation*}


There are five 1D kernels to test here, as follows:

\begin{equation}
a = 1, \Phi(x,\xi) = x\cdot\xi + c(x)|\xi|, c(x) = (2+\sin(2\pi x))/8,
\end{equation}

\begin{equation}
a = 1, \Phi(x,\xi) = x\cdot\xi + c(x)\xi, c(x) = (2+\sin(2\pi x))/5.94,
\end{equation}

\begin{equation}
a = \sum\limits_{k=0}^{n_{k}} e^{-\frac{(x-x_{k})^2 + (\xi-\xi_{k})^2}{\sigma^2}}, \sigma = 0.05, \Phi(x,\xi) = x\cdot\xi + c(x)|\xi|, c(x) = (2+\sin(2\pi x))/8,
\end{equation}

\begin{equation}
a = \sum\limits_{k=0}^{n_{k}} e^{-\frac{(x-x_{k})^2 + (\xi-\xi_{k})^2}{\sigma^2}}, \sigma = 0.1, \Phi(x,\xi) = x\cdot\xi + c(x)|\xi|, c(x) = (2+\sin(2\pi x))/8,
\end{equation}

\begin{equation}
a = \sum\limits_{k=0}^{n_{k}} e^{-\frac{(x-x_{k})^2 + (\xi-\xi_{k})^2}{\sigma^2}}, \sigma = 0.04, \Phi(x,\xi) = x\cdot\xi + c(x)|\xi|, c(x) = (2+\sin(2\pi x))/7,
\end{equation}


\textbf{Note that the amplitude function $a$ in (3), (4), and (5) are as the same as that in Example 2 in Lexing's preprint. Here we skip the exact formula of $a$.}

Discretizing $x$ and $\xi$ on $[0,1)$ and $[-N/2, N/2)$ with $N$ points,
\begin{equation*}
x_{i} = (i-1)/N, \xi_{j} = j-1-N/2.
\end{equation*}
leads to the discrete system $u = Kf$.

\textbf{Regularization.} Here we consider the following L1 optimization problem:
\begin{equation*}
\min\limits_{f} |\Phi(f)| + \|Kf-u\|_{2}^{2},
\end{equation*}
which is equal to 
\begin{equation*}
\min\limits_{f,d} |d| + \|Kf-u\|_{2}^{2} \quad such \quad that \quad d = \Phi(f).
\end{equation*}
To solve this, first convert it into an unconstrained problem:
\begin{equation*}
\min\limits_{f,d} |d| + \|Kf-u\|_{2}^{2} + \frac{\lambda}{2}\|d - \Phi(f)\|_{2}^{2}.
\end{equation*}

This could be solved using the Split Bregman Iteration:
\begin{equation}
\begin{aligned}
Step 1: f^{k+1} &= \min\limits_{f} \|Kf-u\|_{2}^{2} + \frac{\lambda}{2}\|d^{k}-\Phi(f)-b^{k}\|_{2}^2 \\
Step 2: d^{k+1} &= \min\limits_{d} |d| + \frac{\lambda}{2}\|d-\Phi(f^{k+1})-b^{k}\|_{2}^2 \\
Step 3: b^{k+1} &= b^{k} + \Phi(f^{k+1}) - d^{k+1}
\end{aligned}
\end{equation}
The iteration stops when $\|f^{k+1}-f^{k}\| < tol$, and $tol = 1E-10$ in the experiments. $\lambda = 1E-8$. And the real $f$ is set to be sparse in the experiments. 

In Step 2, we can
explicitly compute the optimal value of d using shrinkage operators. We simply compute 
\begin{equation}
d_{j}^{k+1} = shrink(\Phi(f)_{j}+b_{j}^{k}, 1/\lambda),
\end{equation}
where 
\begin{equation}
shrink(x, \gamma) = \frac{x}{|x|}\times \max(|x|-\gamma,0).
\end{equation}

In Step 1, PCG or Gauss-Seidel (GS) method could be used. 

\textbf{ L1 norm: $\Phi(f) = f$.} Then the regularizer is $|f|$, and Step 1 becomes:
\begin{equation}
(2 K^* K + \lambda) f^{k+1} = 2 K^* u + \lambda(d^{k}-b^{k})
\end{equation}
In the experiments, this is solved using PCG with an approximation of $(2 K^* K + \lambda)^{-1}$ as a preconditioner. And an approximation of $(K^* K)^{-1}K^* u$ computed using the HIF idea in Ying's preprint is used to initialize both $f^{0}$ and $d^{0}$. $b^{0}$ is initialized as a zero vector.

\textbf{ TV norm: $\Phi(f) = \nabla f$.} Since $f$ is from the frequency domain (like $[-N/2, N/2]$), then it is not possible to compute its derivatives using finite difference method. To overcome this, we could tranform the frequency domain to the space domain (like $[0,1]$) using FFT, i.e., $f = F g$ where $F$ is the discrete Fourier tranform. Now we should solve $KFg = u$. Then the optimization problem becomes:
\begin{equation*}
\min\limits_{g,d} |d| + \|KFg-u\|_{2}^{2} + \frac{\lambda}{2}\|d - \nabla g\|_{2}^{2},
\end{equation*}
and Step 1 becomes:
\begin{equation}
(2 F^* K^* K F - \lambda \Delta) g^{k+1} = 2 F^* K^* u + \lambda \nabla (d^{k}-b^{k}).
\end{equation}

In the experiments, this is solved using PCG with $(2 F^* K^* K F )^{-1}$ as a preconditioner and $g^{k}$ as an initial guess. We could also compute  $L + U = 2 F^* K^* K F - \lambda \Delta$ using HODLR representation such that $L^{-1}$ and $U$ could be applied quickly. Then Gauss-Seidel method could be applied to solve the equation for $g^{k+1}$. And an approximation of $(F^* K^* K F)^{-1}K^* F^* u$ computed using the HIF idea in Ying's preprint is used to initialize both $g^{0}$ and $d^{0}$. $b^{0}$ is initialized as a zero vector.

Table \ref{1d-k1} summarizes the results for 1D kernel (1).
Table \ref{1d-k2} summarizes the results for 1D kernel (2).
%Table \ref{1d-k3} summarizes the results for 1D kernel (3).
%Table \ref{1d-k4} summarizes the results for 1D kernel (4).
Table \ref{1d-k5} summarizes the results for 1D kernel (5).



%\begin{table}[!htbp]
%\centering
%\begin{tabular}{|c|c|c|c|c|c|}
%\hline
%N & Kernel 1 & Kernel 2 & Kernel 3 & Kernel 4 & Kernel 5\\ 
%
%
%\end{tabular}
%
%\caption{Numerical comparison between HIF and HQR. We solve 1D kernel (1) equation by using the approximate inverse $\hat{G}\hat{K}^{*}$ as preconditioners for PCG with tolerance $1e-8$. We also solve the equation by pure CG without any preconditioners and set the maximum iteration number to be 200.}
%\label{1d-cond}
%\end{table}



\begin{table}[!htbp]
\centering
\begin{tabular}{|c|c|c|c|c|c|c|}
\hline
N & cond &Kernel 1 & Kernel 2 & Kernel 3 & Kernel 4 & Kernel 5 \\ 
\hline
$2^{8}$ & $A$ & 1.0660e+02 & 3.4571e+02 & 5.8246e+02 & 2.6794e+02 & 1.6771e+03\\
~ & $A^{*}A$ & 1.1364e+04 & 1.1952e+05 & 3.3926e+05 & 7.1790e+04 & 2.8128e+06\\
\hline
$2^{9}$ & $A$ & 1.1372e+02 & 6.7118e+02 & 3.7644e+02 & 1.6732e+02 & 3.0517e+03\\
~ & $A^{*}A$ & 1.2932e+04 & 4.5048e+05 & 1.4171e+05 & 2.7995e+04 & 9.3131e+06\\
\hline
$2^{10}$ & $A$ & 1.0870e+02 & 3.9350e+03 & 4.4032e+02 & 1.8616e+02 & 3.3981e+03\\
~ & $A^{*}A$ & 1.1815e+04 & 1.5484e+07 & 1.9388e+05 & 3.4657e+04 & 1.1547e+07\\
\hline
$2^{11}$ & $A$ & 1.1999e+02 & 5.2064e+05 & 4.3108e+02 & 1.9745e+02 & 5.0709e+03\\
~ & $A^{*}A$ & 1.4398e+04 & 2.7107e+11 & 1.8583e+05 & 3.8988e+04 & 2.5714e+07\\
\hline
$2^{12}$ & $A$ & 1.2626e+02 & 1.3755e+10 & 5.5073e+02 & 1.9459e+02 & 3.4614e+03\\
~ & $A^{*}A$ & 1.5943e+04 & 2.9863e+17 & 3.0330e+05 & 3.7866e+04 & 1.1981e+07\\



\end{tabular}

\caption{Condition numbers of all kernels}
\label{cond}
\end{table}




\begin{table}[!htbp]
\centering
\begin{tabular}{|c|c|c|c|c|c|c|c|c|c|c|c|}
\hline
\multicolumn{1}{c|}{} & \multicolumn{1}{c|}{$\hat{K} \approx K$} & \multicolumn{4}{c|}{$HIF$} & \multicolumn{2}{c|}{$L1$} &\multicolumn{2}{c|}{$TV(PCG)$} & \multicolumn{2}{c|}{$TV(GS)$} \\
\hline
N & $e_{a}$ & $e_{h}$ & $\epsilon$ & $e_{f}$ & $e_{s}$ & $n_{i}$ & $e$  & $n_{i}$ & $e$ &  $n_{i}$ & $e$ \\ 
\hline
$2^{8}$ & 9.4e-16 & 1.3e-08 & 1e-7 & 1.0e-08 & 1.5e-08 & 2 & 1.4e-08 & 80 & 3.5e-07 & 80 & 3.5e-07\\
\hline
$2^{9}$ & 1.4e-15 & 1.0e-08 & 1e-7 & 1.1e-08 & 1.1e-08 & 2 & 1.6e-08 & 80 & 3.6e-07 & 80 & 3.6e-07\\
\hline
$2^{10}$ & 1.5e-15 & 1.0e-08 & 1e-7 & 7.7e-09 & 1.5e-08 & 2 & 1.0e-08 & 80 & 3.4e-07 & 80 & 3.4e-07\\
\hline
$2^{11}$ & 2.4e-10 & 9.4e-09 & 1e-7 & 9.2e-09 & 1.1e-08 & 2 & 1.3e-08 & 80 & 3.3e-07 & 80 & 3.3e-07\\
\hline
$2^{12}$ & 3.6e-10 & 9.8e-09 & 1e-7 & 7.7e-09 & 1.2e-08 & 2 & 1.0e-08 & 80 & 2.8e-07 & 80 & 2.8e-07\\
\hline
$2^{13}$ & 1.9e-10 & 8.7e-09 & 1e-7 & 1.2e-08 & 9.8e-09 & 2 & 1.6e-08 & 80 & 3.5e-07 & 80 & 3.5e-07\\
\hline
$2^{14}$ & 1.9e-10 & 9.1e-09 & 1e-7 & 8.4e-09 & 1.2e-08 & 2 & 1.1e-08 & 80 & 3.3e-07 & 80 & 3.0e-07\\
\hline
$2^{15}$ & 5.1e-10 & 9.2e-09 & 1e-7 & 8.2e-09 & 1.2e-08 & 2 & 1.1e-08 & 80 & 3.0e-07 & 80 & 3.8e-07\\
\hline
$2^{16}$ & 4.6e-10 & 9.2e-09 & 1e-7 & 9.5e-09 & 1.1e-08 & 2 & 1.3e-08 & 80 & 3.0e-07 & 80 & 3.5e-07\\

\end{tabular}

\caption{Numerical comparison between different regularizers for kernel (1). $\lambda = 1E-8$. The iteration stops when $\|f^{k+1}-f^{k}\| < tol=1E-10$.}
\label{1d-k1}
\end{table}


\begin{table}[!htbp]
\centering
\begin{tabular}{|c|c|c|c|c|c|c|c|c|c|c|c|}
\hline
\multicolumn{1}{c|}{} & \multicolumn{1}{c|}{$\hat{K} \approx K$} & \multicolumn{4}{c|}{$HIF$} & \multicolumn{2}{c|}{$L1$} &\multicolumn{2}{c|}{$TV(PCG)$} & \multicolumn{2}{c|}{$TV(GS)$} \\
\hline
N & $e_{a}$ & $e_{h}$ & $\epsilon$ & $e_{f}$ & $e_{s}$ & $n_{i}$ & $e$  & $n_{i}$ & $e$ &  $n_{i}$ & $e$ \\ 
\hline
$2^{8}$ & 9.1e-16 & 2.8e-08 & 1e-7 & 1.8e-08 & 1.1e-07 & 2 & 9.7e-08 & 80 & 4.5e-07 & 80 & 4.5e-07\\
\hline
$2^{9}$ & 1.5e-15 & 1.2e-07 & 1e-7 & 1.1e-07 & 2.9e-07 & 2 & 8.2e-07 & 80 & 9.0e-07 & 80 & 7.6e-07\\
\hline
$2^{10}$ & 1.6e-15 & 4.8e-08 & 1e-7 & 1.1e-07 & 2.1e-05 & 2 & 2.9e-04 & 80 & 3.1e-04 & 80 & 1.5e-04\\
\hline
$2^{11}$ & 4.4e-10 & 6.3e-08 & 1e-7 & 1.2e-07 & 5.5e-02 & 1 & 2.7e+00 & 1 & 8.3e+00 & 80 & 8.3e+00\\
\hline
$2^{12}$ & 4.0e-10 & 9.6e-08 & 1e-7 & 1.1e-07 & 4.1e-02 & 1 & 1.6e+01 & 1 & 7.6e-01 & 80 & 7.6e-01\\
\hline
$2^{13}$ & 1.8e-10 & 2.0e-05 & 1e-7 & 1.4e-05 & 5.9e-02 & 2 & 8.9e-01 & 1 & 9.1e-01 & 80 & 1.9e+00\\
\hline
$2^{14}$ & 2.2e-10 & 2.8e-01 & 1e-7 & 1.9e-01 & 4.7e+00 & 2 & 1.3e+01 & 1 & 7.2e+00 & 80 & NaN\\
\hline
$2^{15}$ & 5.5e-10 & 6.3e-01 & 1e-7 & 5.4e-01 & 5.9e+01 & 1 & 7.9e+01 & 1 & 2.0e+01 & 80 & NaN\\
\hline
$2^{16}$ & 5.0e-10 & 6.5e-01 & 1e-7 & 4.3e-01 & 1.9e+01 & 1 & 1.0e+02 & 80 & 4.9e+01 & 80 & NaN\\

\end{tabular}

\caption{Numerical comparison between different regularizers for kernel (2). $\lambda = 1E-8$. The iteration stops when $\|f^{k+1}-f^{k}\| < tol=1E-10$.}
\label{1d-k2}
\end{table}


\begin{table}[!htbp]
\centering
\begin{tabular}{|c|c|c|c|c|c|c|c|c|c|c|c|}
\hline
\multicolumn{1}{c|}{} & \multicolumn{1}{c|}{$\hat{K} \approx K$} & \multicolumn{4}{c|}{$HIF$} & \multicolumn{2}{c|}{$L1$} &\multicolumn{2}{c|}{$TV(PCG)$} & \multicolumn{2}{c|}{$TV(GS)$} \\
\hline
N & $e_{a}$ & $e_{h}$ & $\epsilon$ & $e_{f}$ & $e_{s}$ & $n_{i}$ & $e$  & $n_{i}$ & $e$ &  $n_{i}$ & $e$ \\ 
\hline
$2^{8}$ & 9.0e-16 & 1.3e-08 & 1e-7 & 1.8e-08 & 7.8e-06 & 2 & 1.2e-04 & 80 & 4.9e-03 & 80 & 1.1e-04\\
\hline
$2^{9}$ & 1.2e-15 & 2.0e-08 & 1e-7 & 3.1e-08 & 1.6e-05 & 2 & 1.7e-04 & 80 & 2.0e-03 & 80 & 1.1e-04\\
\hline
$2^{10}$ & 1.3e-15 & 3.3e-08 & 1e-7 & 5.0e-08 & 2.6e-05 & 2 & 3.3e-05 & 80 & 3.1e-03 & 80 & 2.6e-04\\
\hline
$2^{11}$ & 2.7e-10 & 2.5e-08 & 1e-7 & 2.6e-08 & 8.8e-06 & 2 & 7.2e-05 & 80 & 2.6e-03 & 80 & 2.6e-04\\
\hline
$2^{12}$ & 2.7e-10 & 3.5e-08 & 1e-7 & 3.9e-08 & 3.5e-06 & 2 & 8.7e-06 & 80 & 2.9e-03 & 80 & 3.9e-04\\
\hline
$2^{13}$ & 2.1e-10 & 3.6e-08 & 1e-7 & 3.7e-08 & 8.8e-06 & 2 & 2.5e-05 & 80 & 1.7e-03 & 80 & 5.0e-04\\
\hline
$2^{14}$ & 2.1e-10 & 3.3e-08 & 1e-7 & 4.4e-08 & 7.6e-06 & 2 & 7.1e-05 & 80 & 2.1e-03 & 80 & 6.6e-03\\
\hline
$2^{15}$ & 5.2e-10 & 3.1e-08 & 1e-7 & 3.6e-08 & 4.2e-05 & 2 & 1.4e-04 & 80 & 2.8e-03 & 80 & 2.6e-02\\
\hline
$2^{16}$ & 5.1e-10 & 3.3e-08 & 1e-7 & 2.9e-08 & 1.4e-05 & 2 & 1.2e-05 & 80 & 2.0e-03 & 80 & 1.7e-01\\



\end{tabular}

\caption{Numerical comparison between different regularizers for kernel (5). $\lambda = 1E-8$. The iteration stops when $\|f^{k+1}-f^{k}\| < tol=1E-10$.}
\label{1d-k5}
\end{table}









\bibliographystyle{unsrt} 
\bibliography{ref}

\end{document}

